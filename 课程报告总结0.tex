                                               课程总结报告                 
  一.引言
   随着当今时代科技快速发展,计算机也经历了种种变迁,每一代计算机的变化都是在众多科学家无数的智慧结晶的基础上所得到的,从第一代机械型的计算机直到现在的第五代智能化计算机,计算机在日常生活和国家的科学发展中都密不可分。如今的计算机在全世界的普及率越来越高,可见人们对计算机的重视程度。这是一个科技普及速度飞快,信息爆炸的年代,计算机起着决定性的作用,学生们都对计算机有一些了解,我正是通过我的了解,对计算机有了浓厚的兴趣,才报选了计算机科学与技术这门专业,我了解到了计算机史的艰辛,了解到计算机的组成,了解到计算机的运行,了解到计算机的衍生,这些无一不激起我对计算机的热爱,而计算科学导论这门课程为我敲开了专业之门,我亦因之的收获颇深。下面是我对计算科学导论的认识和我的所得。
   二.对计算科学导论这门课程的认识、体会
    经过了差不多一个学期的课程学习,我已经有了一些对于计算机的一些认识和体会,计算科学导论就像是一本书的目录,它涵盖了许多关于计算机的名词和概念,可以让我们对于计算机科学与技术这门专业有一个全面的初步判断,并且可以为我们更好的完成学业作参考,尽管许多知识对于我来说是非常的枯燥且乏味,让人感觉到苦涩,但当我可以静下心来慢慢品味的时候,我对于这门学科又有了更深刻的理解,不是所有的内容都是易懂的,而且有一些知识的概念也不是很详细,导致同学们对于一些深入的内容理解起来有些困难,但同学们大致还是可以获得大概的意思和某些内容的初步了解,只要肯努力,进一步了解也并不是什么难事。
   学校对于计算科学导论这门学科也给予了很大的重视,一周4节课的频次代表着它的地位。在我看来,计算科学导论是对于计算机科学与技术专业所有知识及教学内容的一个简洁明了的概括,准确来说的话,应该是压缩,它把许多知识以一种简单的语言来表达出来,让同学们可以清楚的学习到一个名词。而且,在大一就开设了这门课程,可以让同学们对于计算机科学与技术有所了解,通过概括的方式让同学们学习到更多的内容,而且可以解答同学们的一些疑惑,对自己以后未来的职业规划有极大的帮助和指导。计算科学导论就学科特点、学科形态、历史渊源、发展变化、典型方法、学科知识组织结构和分类体系、各年级课程的重点,以及如何认识计算科学,学好计算科学等问题从科学哲学和高级科普的角度去回答大家疑问。
   如果仅仅以为学习这门课程就可以不再学习那可就大错特错了,这门课程知识给了同学们一条方向,但是更加深入的东西,更加基础的东西,还需要同学们积极的汲取,学习。这门课程的内容缺乏实际的科学论证,还需要搭配后续课程的知识,但是这门课程所占学时数的比例有很大,也导致了增加一些本不属于这个专业的教学内容,以便我们升入高年级打好基础。
    作为计算机科学与技术的一名学生,我对计算机的兴趣颇深,我也充满了信心和斗志,而计算科学导论这门课程就像是一把钥匙,帮我打开了专业之门,让我可以随意汲取之中的知识,使我受益匪浅。
    目前我所学到的关于计算机的课程有英语,高等数学,离散数学,这些课程都和计算机有许多关系。这些课程并不是所有,但绝对是基础,只有把这些课程学明白,才可以为以后打下基础。数学在生活中的用处有很多,但对我来说,最主要的是,可以帮助我更好的学习计算机相关内容。除此之外,英语也很重要,众所周知,英语是人与计算机交互的基本语言,世界上的许多先进高等的资料文献都是由英语来撰写的,所以学习英语必不可少。作为计算机科学与技术的一名普通学生,我不仅要学习基本的英语,还得参加并通过英语四级考试,注重英语在计算机方面的重要性,以便可以促进我对专业的理解和对计算机知识的学习。所有我学的课程就像是一个木桶所需组成的木板,每一块木板都不可以缺少,否则就会产生很大的影响。
    通过近一个学期的学习,我对计算机的发展史等有了许多新的认识与理解。例如,20世纪30年代是计算模型取得突破进展的时期,当时的许多人,如哥德尔,丘奇,图灵,波斯特等人在研究中陆续提出了一批计算模型,为后来的的计算机打下了基础。其中图灵机最为重要,在图灵机提出后不到十年,世界上第一台储存程序式通用电子数字计算机就诞生了,由于图灵的杰出贡献,ACM每天都是表彰对计算科学作出卓越贡献的科学家颁发图灵奖。而且美国信息交换标准代码(ASCII)也已被大多数国家所采纳。除此之外,我也了解到了计算机的基本概念和基本知识,尽管不同书籍对于某些定义会有所区别,但大体内容还是一样的,都极尽详细的描写,让我们可以清楚这些知识。在此之前,我还可笑的认为对于计算机有贡献的只有比尔盖茨一个人,却不知有着许多大牛为了计算机做出了许多卓越的贡献,他们在计算机发展史上所有的地位是不可取代了,是无法忽视的,他们都是伟人,都是巨人。我从课本中了解到了二进制,这个耳熟能详却又很陌生的词语,经过我的了解,发现二进制真的很牛,真的很佩服先人们,这么聪明的成果得付出了多少努力啊,而且我还学习到了八进制,十六进制,三十二进制等等,这些在我了解二进制后都非常简单了。这么多的知识,有困难的知识,也有简单的知识,都围绕在我身边,真的是让我又爱又恨,深深的引起了我的兴趣。我还接触到了布尔代数基础,通过书中的解释,我了解到了布尔代数的公理系统和标准形式及公理系统的完备性,还了解到了许多关于布尔代数的定理,许多新的知识都让我充满了新鲜感,让我动力满满。   
    当我从书中看到了布尔代数,我便对这个新鲜的词产生了兴趣,布尔代数对于计算机的作用大不大呢,我对于这个问题上网查询,我得知布尔代数是计算机的基本运算方式,因为布尔代数只有两个基本元素:1(ture,真)和0(false,假),三种基本运算:与,或,非。看到这里我正好想到了计算机的数字电路的原理,两者不都是把数据转化成1或者0来运输数据的么,那么两者肯定有一定的关系。布尔代数把数据通过转化,把数据都简单化了,在通过计算机运输,布尔代数把所有东西都符号化,完全代数化了,计算机正是通过这点,把大量数字电路组合起来,同时运行无数个布尔运算,才有了速度过人的结果,可见,布尔代数对计算机的作用至关重要,没有布尔代数,几乎也就没有了如今的计算机。
   三.进一步的思考
    通过我的分组演讲报告题目“智能客服",我了解到了许多关于智能客服的内容,智能客服是经过数十年的科技进步,在大数据时代的背景下所诞生的产物。客服这个概念最早是1956年有泛美航空公司推出客服中心,应用于客户机票预定所诞生的,随后客服由电话的形式全球推广开来,之后不断改进,才有了今天的智能客服。智能客服是在大规模知识处理基础上发展起来的一项面向行业应用的,它是(大规模知识处理技术、自然语言理解技术、知识管理技术、自动问答系统、推理技术等等),具有行业通用性,不仅为企业提供了细粒度知识管理技术,还为企业与海量用户之间的沟通建立了一种基于自然语言的快捷有效的技术手段;同时还能够为企业提供精细化管理所需的统计分析信息。简单的来说,智能客服不仅可以进行客户与公司的交流,还可以反馈给公司数据,以便公司可以优化和改进产品。智能客服系统需要有必须的功能组成,如语音识别技术、语音合成及NLP等基本的AI工作能力,还需具备精确的语义检索能力,不断更新语句知识库,以便应对更复杂的交流环境,并且话务员可以在线编辑知识库,供其他话务员使用,或者经过审核后,供智能客服系统自动使用。智能客服机器人可以24可透过一定的载体,如WEB,IM,WAP/SMS等结合图片,文字甚至音视频等媒体给用户最完整的回复,让用户在交流中解决问题。如今的时代飞速发展,许多的内容也充实着这个网络世界,所以具有自学能力也是需要的,所以有智能客服系统和机器人所组合成的智能机器人,它具备自学能力,可以通过网络从互联网上自主的获取所需的知识,来充实自己的文本库,随着不断的进行网上学习,来持续的扩大累计的知识,进而推动智能客服系统不断提升,达到节省人力资源的目的。
    客服是连接企业与客户的重要桥梁,不仅影响着客户的心情,也影响着企业的销售成果,品牌口碑和市场地位,更可能影响企业的未来,所以作为企业客户瓜西管理的重要组成部分,客服至关重要。中国的客服软件市场大致由三个发展阶段组成,传统呼叫中心软件,PC网页在线客服+传统客服软件,云客服+客服机器人的智能客服阶段。
     基本的人机交互技术是不能满足如今攻公司多变的需要,智能客服系统必须要具有基本的产品专业知识,而且可以满足客户的诸多要求,如果搭配上了语音功能的话,就可以及时的抚慰用户的心情,再加上真人语音系统,就可以给客户一个完美的使用体验。
    通过智能客服系统,公司会有诸多好处,如减轻客服压力,降低运营成本,提高营销能力,规范行业知识库,树立服务形象,提升满意度。除此之外,还可通过大数据,综合处出多个典型问题,方便用户查找,节省了客户的时间和公司的人力资源,对双方都有好处。相比较于传统的客服,如人工服务,无法提供1对1服务体验,无法7*24小时一直在线,需要大量的岗前培训,人员流动率高,管理成本高,而且还有员工跳槽的风险;还有机器人客服,机器人难以解决复杂问题,只是机械式的反馈特定的答案,并且知识整理工作量大。种种缺点,都代表着传统客服的难以广泛适用性,所以如今的智能客服随之诞生。早期客服机器人的出现一定程度上解决了简单重复性问题,而深度学习算法的应用又降低了客服机器人所依赖的知识库构建和维护成本的大幅下降,加上大数据分析和智能技术在客服场景深入应用,AI正在变革客服行业的原有业态。智能客服系统支持将不同的业务分流到不同的机器人接待,可以减轻服务器和人员压力,当访客在排队时,机器人可以率先接待,来减少排队时间,也支持语音识别,访客可以通过语音录入直接向机器人提问,机器人可以通过情绪识别,当访客的心情是愤怒或者焦虑的时候,可以第一时间安抚客服,并且立即转入人工服务。如今的机器人可以支持复杂场景对话,可以随时转入人工,可以在机器人答案或任何可见入口,设置流畅的转人工较强的语义理解能力,偏差不超过2%在人工接待时,机器人也可以辅助快速回复。
   智能客服具有行业适用性,大部分的行业都需要客服,所以这个系统就不可缺少,它所衍生出的功能也有很多,除了最基本的客服系统外,还如云电销,云呼叫中心,智能语音机器人,工作手机,大屏监控,CRM,云视频会议,BI智能报表和大数据能力。所使用的场景也有很多种,如获客+服务,绩效+考核,智能问答服务,语音自助服务,客户回访,拓客营销,行为监控,数据管理。公司可以通过智能客服系统所回馈的大数据来整理出一定的规律,来不断的调节自己的系统性能。现如今中国的客服软件市场非常激烈,抢夺的重点不在云客服厂商所擅长的互联网领域,而在于传统行业,因为互联网服务行业较浅,且大公司自研客服系统,中小企业付费能力较弱,不足以将云客服厂商养成巨头。而对传统行业大客户的争夺一方面靠过硬的技术和产品实力,另一方面还要依赖资源和服务水平,这对于才发展几年的云客服厂商来说挑战重重,我国的客服软件还有很长的一段里要走。
    综合从我就我所了解的知识来说,智能系统大致可以有四个方面:智能沟通,智能分析,精准执行,智能学习。 同时,我不禁思考,在未来时候,是不是可以存在完全的机器人客服,不再需要人工服务了,因为那时数据库的知识足以应对所有的复杂对话,而人类又何去何从呢?就目前来说,客服机器人正在以40-50%的比例代替人工客服工作,如今中国大约由500万全职客服,但在未来还剩多少人谁也不知道,但一定是很少就对了,哪会有多少人因此下岗呢?
    通过这次的演讲准备,我又了解到了许多关于计算机的知识,我也对智能客服这方面又一点兴趣,感觉我可以深入的了解一下,尝试一下这方面的内容,趁着我还年轻,我要去试试,挑战一下自己。
    四.总结
   我觉得,只是学习课本的知识是远远不够的,还需要通过自己的双手去实践,去检验,这样才可以更好的学习到知识。通过许多前人的经验,我了解到,这门专业的学习不是简单的,它需要我们一天天的积累,一天天的学习,因为这门专业需要许多的实践经验,不像是语文英语那种学科一样,通过记忆背诵就可以取得成果的。时代发展飞速,一些计算机语言也随之诞生,我们即使可以一直用我们原来的,但是新的语言的学习也是不可缺少的,学习到更多的知识,对自己现有的知识也是一种巩固。在计算机科学与技术这门领域上,有这许多伟人,他们经历了许多崎岖的,充满荆棘的人生之路,最终获得的巨大成果。既然选择了这条道路,就必须为之奋斗,不离不弃,一生为之奉献。
   对于我这个对计算机充满了期待,但是有是个刚刚接触计算机的小白来说,一切感觉起来都很遥远,现在我接触的有关计算机内容的只有计算科学导论和一本c++的书,但我对一切都很好奇,我觉得我可以在计算机这条道路上走得更远。原来我以为计算机是很神奇的,也是很复杂的,但当我学习后,觉得一切也不过是这样啊,计算机不管什么类型,硬件结构都包括运算器,控制器,储存器,输入设备和输出设备,但是计算机如果没有软件系统的支持的话,就像是手机没有一样,如同一堆废铜烂铁。学习了这么多知识,计算机那个神秘的面纱也缓缓摘下,露出它那美丽的容貌,让人陶醉沉迷其中,无法自拔。
  如今这个时代,计算机已经成为了生活中必不可少的部分,计算机科学与技术专业在全国范围内都已经出现了加速的趋势,学习计算机迫在眉睫,经济的快速发展和外来生活方式的文化的影响让我们不得已通过互联网来连接起全世界,以免造成一些知识的混乱。我通过学习,了解计算机,希望可以用计算机来解决生活中问题,计算机专业的实用性很高,生活中的许多事情都可以用计算机来解决。在我学习计算机前,我是迷惘的,但是现在,我有了自己的目标,我先成为一名程序员,可以为国家的网络做出一份贡献,有了一个目标,人才有了前进的动力,并不断为之奋斗。而不是混日子,等着得到一张毕业证去到处找工作。我给自己定下目标,要学好高数,离散数学,大学物理和线性代数这些和计算机有关的学科,对学习认真,对学习负责,对自己负责。
  五.参考文献
  中国智能客服行业研究报告
  https://wenku.baidu.com/view/7f28cf1ea55177232f60ddccda38376baf1fe02a.html
  鲸准研究院:2018中国智能客服行业研究报告
  http://www.199it.com/archives/730844.html
  智能客服机器人工作原理 
   http://www.huayunworld.com/content/152
   计算机发展史及未来前景论文
   https://www.docin.com/p-217977315.html
   为什么说布尔代数是计算机的基本运算方式
    http://www.360doc.com/content/17/0125/17/29351439_624740064.shtml
   六.附录
  
